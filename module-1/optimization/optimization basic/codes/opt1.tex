\documentclass[a4paper,10pt]{report}
\usepackage[latin1]{inputenc}
\usepackage{amsmath}
\usepackage{amsmath,bm}
\usepackage{amsthm}
\usepackage{mathtools}
\usepackage{amsfonts}
\usepackage{amssymb}
\usepackage{graphicx}
\usepackage{array}
\usepackage{booktabs}
\usepackage{hyperref}
\usepackage{multicol}
\usepackage{makecell}
\usepackage[margin=0.5in]{geometry}
\usepackage[framemethod=tikz]{mdframed}
\newcommand{\myvec}[1]{\ensuremath{\begin{pmatrix}#1\end{pmatrix}}}
\let\vec\mathbf
\newcommand{\mydet}[1]{\ensuremath{\begin{vmatrix}#1\end{vmatrix}}}
\providecommand{\mbf}{\mathbf}
\providecommand{\pr}[1]{\ensuremath{\Pr\left(#1\right)}}
\providecommand{\qfunc}[1]{\ensuremath{Q\left(#1\right)}}
\providecommand{\sbrak}[1]{\ensuremath{{}\left[#1\right]}}
\providecommand{\lsbrak}[1]{\ensuremath{{}\left[#1\right.}}
\providecommand{\rsbrak}[1]{\ensuremath{{}\left.#1\right]}}
\providecommand{\brak}[1]{\ensuremath{\left(#1\right)}}
\providecommand{\lbrak}[1]{\ensuremath{\left(#1\right.}}
\providecommand{\rbrak}[1]{\ensuremath{\left.#1\right)}}
\providecommand{\cbrak}[1]{\ensuremath{\left\{#1\right\}}}
\providecommand{\lcbrak}[1]{\ensuremath{\left\{#1\right.}}
\providecommand{\rcbrak}[1]{\ensuremath{\left.#1\right\}}}
\begin{document}
\raggedleft FWC22037\vspace{2mm}\\
\centering\Large\textbf{ASSIGNMENT-OPTIMIZATION}\vspace{5mm}
\begin{multicols}{2}
\centering \large\textsc{C}\footnotesize\textsc{ONTENTS}\vspace{5mm}\\
\raggedright\large\textbf{1\hspace{1cm}Problem}\hspace{5.2cm}1\vspace{5mm}\\
\raggedright\large\textbf{2\hspace{1cm}Solution}\hspace{5.25cm}1\vspace{5mm}\\
\raggedright\large\textbf{3\hspace{1cm}Construction}\hspace{4.25cm}2\vspace{5mm}\\
\centering \large\textsc{1  P}\footnotesize\textsc{ROBLEM}\vspace{5mm}\\
	\raggedright\large{A solution of 8\% Boric acid is to be diluted by adding a 2\% Boric acid solution to it.The resulting mixture is to be more than 4\% but less than 6\% Boric acid. If we have 640 liters of the 8\% solution,how many liters of the 2\% solution will have to added ?}\\\vspace{5mm}
\centering \large\textsc{2  S}\footnotesize\textsc{OLUTION}\vspace{5mm}\\
\raggedright\large{1.Consider total amount to be (x+640)liters\\}
\raggedright from the given information \\
\begin{align}
2\% of x + 8\% of 640 > 4\% of (x+640) \\
2\% of x + 8\% of 640 < 6\% of (x+640)
\end{align}
2.From equation 1 we can solve and get the maximum of x\\
\begin{gather*}
\frac{2x}{100} + \frac{8X640}{100}  > \frac{4}{100}(x+640)\\
2x+5120 > 4x+2560\\
2560 > 2x \\
x<1280\\
\end{gather*}
3. From equation 2 we can solve and get the minimum of x\\
\begin{gather*} 
 \frac{2x}{100} + \frac{8X640}{100}  > \frac{6}{100}(x+640)\\
 2x+5120 < 6x+3840\\
 1280<4x\\
 320<x\\
 \end{gather*}
 The x lies in between 320 and 1280.
 \vspace{10mm}


\raggedright\large{The python code provided in the below source code link.} \\
\begin{mdframed}
\raggedright\large{https://github.com/sivagayathri \\ /FWC/blob/main/opt/opt-1.py}
\end{mdframed}
\end{multicols}
\end{document}
